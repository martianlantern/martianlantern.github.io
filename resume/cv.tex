%%%%%%%%%%%%%%%%%%%%%%%%%%%%%%%%%%%%%%%%%
% Medium Length Professional CV
% LaTeX Template
% Version 2.0 (8/5/13)
%
% This template has been downloaded from:
% http://www.LaTeXTemplates.com
%
% Original author:
% Trey Hunner (http://www.treyhunner.com/)
%
% Important note:
% This template requires the resume.cls file to be in the same directory as the
% .tex file. The resume.cls file provides the resume style used for structuring the
% document.
%
%%%%%%%%%%%%%%%%%%%%%%%%%%%%%%%%%%%%%%%%%

\documentclass{resume}
\usepackage{hyperref, textcomp, lipsum}
\usepackage{enumitem}
\usepackage[dvipsnames]{xcolor}
\usepackage{graphicx}
\hypersetup{
    colorlinks=true,
    linkcolor=blue,
    filecolor=blue,      
    urlcolor=blue,
}
\usepackage[left=0.5in,top=0.5in,right=0.6in,bottom=0.5in]{geometry}

\name{Darshan Makwana}
\address{\textit{\href{mailto:darshanmakwana412@gmail.com}{darshanmakwana412@gmail.com}} \\ \href{https://github.com/martianlantern}{GitHub} \\ \href{https://www.linkedin.com/in/darshan-makwana-282480228/}{LinkedIn}}
\address{\textbf{WEBPAGE}: \url{https://martianlantern.github.io}}

\begin{document}

\vspace*{0.1in}
\noindent
\begin{minipage}[t]{0.75\textwidth}
My name is Darshan Makwana and I'm a Machine Learning Engineer at \href{https://www.sprinklr.com/}{Sprinklr} working on LLM Sys and real time voice AI systems. Previously, I was a Research Assistant at \href{https://users.aalto.fi/~kannalj1/}{Aalto Vision Lab} working on 3D Gaussian Splatting and also did research internships at \href{https://www.tum.de/en/}{TU Munich} on quadruped robotics and \href{https://www.hexo.ai/}{Hexo AI} on diffusion models.

\vspace{0.1in}
I graduated from \href{https://www.iitb.ac.in/}{IIT Bombay} with a B.Tech in Mechanical Engineering and Dual Minor in Computer Science and Machine Learning. My interests span computer vision, 3D reconstruction, and efficient ML systems.
\end{minipage}%
\hfill
\begin{minipage}[t]{0.21\textwidth}
\vspace{0pt}
\includegraphics[width=1.1\textwidth]{./../assets/images/darshan1.png}
\end{minipage}
\vspace{0.1in}

\begin{rSection}{Education}
\vspace*{0.1in}

{\bf Indian Institute of Technology, Bombay} \hfill {2021 - 2025} \\ 
B.Tech in Mechanical Engineering $|$ Minor in Computer Science and Data Science\\
Major CPI: 8.82/10 $\cdot$ Minor CPI: 10/10

\begin{itemize}[leftmargin=1.5em, itemsep=0.1em]
\item \textbf{Team Lead, Inter IIT Tech Meet 13.0} \hfill Dec 2024\\
{\small Led team of 13 for Adobe deepfake detection challenge to 2nd runner up in task 2}
\item \textbf{Teaching Assistant} \hfill Apr - Jun 2023\\
{\small PH 112 \& MA 108 $|$ Conducting weekly tutorials and grading assignments for 40+ students}
\item \textbf{Machine Learning Mentor} \hfill May - Jul 2023\\
{\small Mentoring 7 students during the summers to pick up pace in ML}
\item \textbf{Core Team Member, DAV} \hfill Jul 2022 - Apr 2023\\
{\small Data Analytics Team of the Undergrad Academic Council}
\item \textbf{Web Coordinator} \hfill May 2022 - Jun 2023\\
{\small Developed websites for SARC events and alumni portals}
\end{itemize}

\vspace{0.1in}

{\bf Aalto University, Finland} \hfill {Jan - Jun 2025} \\ 
Semester Exchange, School of Science\\
Ranked 3/380+ in Programming Parallel Computers Course\\
\href{https://martianlantern.github.io/assets/pdfs/ppc_lor.pdf}{Letter of Recommendation}\\
CPI: 10.0/10.0

\end{rSection}

\begin{rSection}{Experience}
\vspace*{0.1in}

{\bf Sprinklr} {\hfill Gurugram, India}
\begin{itemize}[leftmargin=1.5em, itemsep=0.1em, topsep=0.1em]
\item \textbf{Voice Hackathon Winner} \hfill Jan 2026\\
{\small Won 2nd prize in company wide voice hackathon. Our team built an end to end voice agent prototype with real time multi-turn conversation handling and tool calling capabilities, competing against 20+ teams.}
\item \textbf{Machine Learning Engineer} \hfill Aug 2025 - Present\\
{\small Building real time voice bot infrastructure. Optimizing STT pipelines with vLLM and with custom scheduler for priority based request routing. Working on end to end voice bots for conversational AI and their deployment pipelines.}
\item \textbf{Product Intern} \hfill May - Jul 2024\\
{\small Benchmarked audio codecs for use in voice agent pipelines. Built a unified voice bot prototype for ASR and TTS applications achieving WER 8.9\%, $\sim$300ms latency. Received pre placement offer (PPO).}
\end{itemize}

\vspace{0.15in}

{\bf Aalto Vision Lab} {\hfill Helsinki, Finland}\\
\textit{Research Assistant under \href{https://maturk.github.io/}{Matias Turkulainen} and \href{https://users.aalto.fi/~kannalj1/}{Juho Kannala}} \hfill Jan - Jul 2025\\
Wrote multi GPU training scripts for feedforward gaussian splatting on SLURM clusters. Profiled and optimized training kernels for better GPU utilization. Benchmarked and ablated various optimization techniques for feedforward gaussian splatting.

\vspace{0.15in}

{\bf Technical University of Munich} {\hfill Munich, Germany}\\
\textit{Summer Research Intern $|$ Quadruped Robotics under \href{https://hp-cao.github.io/hongispage/}{Hongpeng Cao} and \href{https://rtsl.cps.mw.tum.de/personal_page/mcaccamo/}{Marco Caccamo}} \hfill Jun - Jul 2023\\
Developed system identification models for quadruped robot dynamics. Implemented MPC based control and compared against model free deep RL approaches in simulation.

\vspace{0.15in}

{\bf Hexo AI} {\hfill Remote}\\
\textit{ML Intern $|$ Diffusion Models} \hfill Feb - Apr 2023\\
Built a text conditional latent diffusion model for image generation. Worked on image inpainting with textual conditioning in a team of 3.

\vspace{0.15in}

{\bf 64Squares} {\hfill Remote}\\
\textit{ML Intern $|$ Contextual Chatbots} \hfill Dec 2022 - Jan 2023\\
Trained BERT and spaCy models with DIET classifier for intent and entity extraction. Stress tested chatbot models for stability.

\end{rSection}

\begin{rSection}{Projects}
\vspace*{0.1in}

{\bf ThinkMesh} $|$ \href{https://github.com/martianlantern/ThinkMesh}{code} {\small (GitHub $\star$ 274)} \hfill Sep 2025\\
Parallel reasoning framework for LLMs implementing \href{https://arxiv.org/abs/2203.11171}{Self-Consistency}, \href{https://arxiv.org/abs/1805.00899}{Debate}, \href{https://arxiv.org/abs/2305.10601}{Tree of Thoughts}, and \href{https://arxiv.org/abs/2308.09687}{Graph of Thoughts} strategies with configurable compute budgets.

\vspace{0.15in}

\noindent
\begin{minipage}[t]{0.8\textwidth}
{\bf Gaussian Masked Autoencoders} $|$ \href{https://github.com/darshanmakwana412/gaussian-mae}{code} \hfill Feb - Mar 2025\\
Replicated gaussian MAE paper using 3D Gaussian splats as intermediate latent representation, enabling zero-shot capabilities.
\end{minipage}%
\hfill
\begin{minipage}[t]{0.15\textwidth}
\vspace{-5pt}
\includegraphics[width=\textwidth]{../assets/projects/gmae.png}
\end{minipage}

\vspace{0.15in}

\noindent
\begin{minipage}[t]{0.8\textwidth}
{\bf Minimal 2D Gaussian Splatting} $|$ \href{https://github.com/darshanmakwana412/mings}{code} \hfill Dec 2024\\
Implementation of gaussian splatting for 2D images using tile-based differential rasterization with custom Triton kernels.
\end{minipage}%
\hfill
\begin{minipage}[t]{0.15\textwidth}
\vspace{-5pt}
\includegraphics[width=\textwidth]{../assets/projects/toucan.png}
\end{minipage}

\vspace{0.15in}

\noindent
\begin{minipage}[t]{0.8\textwidth}
{\bf Multi-View Reconstruction} $|$ \href{https://github.com/darshanmakwana412/Multi_View_Reconstruction}{code} $|$ \href{https://github.com/darshanmakwana412/Multi_View_Reconstruction/blob/main/report.pdf}{report} \hfill Sep - Nov 2024\\
3D reconstruction from images using volumetric graph cuts on visual hulls followed by ray casting for shading.
\end{minipage}%
\hfill
\begin{minipage}[t]{0.15\textwidth}
\vspace{-5pt}
\includegraphics[width=\textwidth]{../assets/projects/mvs.png}
\end{minipage}

\vspace{0.15in}

\noindent
\begin{minipage}[t]{0.8\textwidth}
{\bf Structural Optimization} $|$ \href{https://github.com/darshanmakwana412/structure_optimization}{code} $|$ \href{https://github.com/darshanmakwana412/structure_optimization/blob/main/main.pdf}{report} \hfill Jan - Apr 2024\\
Gradient-based optimization framework using matrix formulation to compute optimal spatial configuration of structures under loading.
\end{minipage}%
\hfill
\begin{minipage}[t]{0.15\textwidth}
\vspace{-5pt}
\includegraphics[width=\textwidth]{../assets/projects/struct_optim.png}
\end{minipage}

\vspace{0.15in}

\noindent
\begin{minipage}[t]{0.8\textwidth}
{\bf Fractal Curves for Tool Path Planning} $|$ \href{https://github.com/darshanmakwana412/fractal}{code} $|$ \href{https://github.com/darshanmakwana412/fractal/blob/main/draft/tool_path_planning.pdf}{report} \hfill Aug - Nov 2023\\
Fractal curve algorithms for tool path planning in layered 3D printing with recursive decomposition.
\end{minipage}%
\hfill
\begin{minipage}[t]{0.15\textwidth}
\vspace{-5pt}
\includegraphics[width=\textwidth]{../assets/projects/fractal.png}
\end{minipage}

\end{rSection}

\begin{rSection}{Publications}
\vspace*{0.1in}

{\bf COVID-19 Self diagnosis classification using BERT and LightGBM models}: Shayona \hfill Jul'23\\
R Chavda, \textbf{Darshan Makwana}, et al. Accepted at SMM4H 2023 (Social Media Mining for Health Applications)

\end{rSection}

\end{document}
